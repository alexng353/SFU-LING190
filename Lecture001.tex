\documentclass[12pt]{article}
\usepackage{amsmath}
\usepackage{amssymb}
\usepackage{amsthm}
\usepackage{amsfonts}
\usepackage{graphicx}
\usepackage{textcomp}
\usepackage{hyperref}
\usepackage{tikz}
\usepackage{enumitem}
\usepackage{mathtools}
\usepackage{enumitem}
\usepackage{wasysym}
\usepackage{ulem}
\usepackage{xspace}
\usepackage{booktabs}
\usepackage{physics}

\ifPDFTeX % ensure generation of machine readable output
\input{glyphtounicode}
\pdfgentounicode=1
\usepackage[T1]{fontenc}
\usepackage[utf8]{inputenc}
\usepackage{lmodern}
\fi

\usepackage{csquotes}

\DeclareMathOperator{\dist}{dist}
\DeclareMathOperator{\Nul}{Nul}
\DeclareMathOperator{\Row}{Row}
\DeclareMathOperator{\proj}{proj}

\setlength{\arraycolsep}{12pt}

\newcommand{\Eg}{\textbf{Eg.}\xspace}
\newcommand{\Ex}{\textbf{Ex.}\xspace}
\newcommand{\Ie}{\textbf{I.e.}\xspace}
\newcommand{\bigEps}{\mathcal{E}}
\newcommand{\bproof}{\textit{Proof ($\impliedby$).}\xspace}
\newcommand{\defn}{\textbf{Def.}\xspace}
\newcommand{\eg}{\textbf{e.g.}\xspace}
\newcommand{\ex}{\textbf{ex.}\xspace}
\newcommand{\fproof}{\textit{Proof ($\implies$).}\xspace}
\newcommand{\ie}{\textbf{i.e.}\xspace}
\newcommand{\lemma}{\textit{Lemma}\xspace}
\newcommand{\soln}{\textit{Soln.}\xspace}
\newcommand{\thm}{\textbf{Thm.}\xspace}

\renewcommand{\arraystretch}{1.25} % Adjust row spacing

\hypersetup{
    colorlinks=true,
    linkcolor=blue,
    filecolor=blue,      
    urlcolor=blue,
}

\newcommand{\ulhref}[2]{\href{#1}{\color{blue}\uline{#2}}}

\begin{document}

\title{LING 190 Lecture 0 - Introduction}
\author{Alexander Ng}
\date{May 12, 2025}

\maketitle

\section{Speech Acoustics}

\subsection{Basic Terminology}

\begin{itemize}
  \item Transverse Waves: Not soundwaves
  \item Longitudinal Waves: Back and forth movement of particles, which is a
    transfer of energy from one particle to another. Motion is linear.
  \item Periodic sounds: Pressure wave of a specific shape is repeated (think
    tone)
  \item Aperiodic: No repeating pattern (think white noise)
  \item Transient Aperiodic Sounds: Aperiodic sounds that last for a short
    time.
    \begin{itemize}
    \item Think the \enquote{ch} sound in \enquote{speech}
    \end{itemize}
\end{itemize}

\subsubsection{Some Lawful Relations}

Wavelength $(\lambda)$ is a measure of distance (in metres)

Period $(P)$ is a measure of time (in seconds)

Frequency $(f)$ is a measure of cycles per second (in Hz, $s^{-1}$)

\[
f = \frac{\text{speed of sound}}{\lambda}
.\]

\[
  \lambda = \frac{\text{speed of sound}}{f}
.\]

\[
  P = \frac{1}{f}
.\]

\subsubsection{Three kinds of information useful for describing periodic sound
waves}

\begin{enumerate}
  \item Frequency: Cycles per second, or Hz, which we perceive as pitch
  \item Amplitude: Displacement of the wave (not the same as intensity or loudness)
  \item Phase: How a wave starts
    \begin{enumerate}
      \item Note: we do not perceive phase
      \item Basically the offset or angle of the wave from the x-axis, at the start
        of the wave.
    \end{enumerate}
\end{enumerate}

\subsection{Frequency}

\begin{itemize}
\item Lower frequency = lower pitch
\item Higher frequency = higher pitch
\item Humans hear a wide range of frequencies, roughly from 20 Hz to 20,000 Hz
\end{itemize}

\subsection{Loudness}

\begin{itemize}
  \item Not the same thing as amplitude
  \item Frequency \textbf{and} amplitude affect loudness (total amount of
    \enquote{energy} in a wave)
  \item Measured in dB (decibels/decibel scale)
  \item deci = 1/10th of a Bel (Named after Alexander Graham Bell)
    \[
      L_p = 20\log_{10}\pqty{\frac{p_{\text{rms}}}{p_{\text{ref}}}} dB
    .\]
  \item This measure is a ratio of sound energy over area (Watts/Metre Squared)
    = \enquote{sound pressure level} (SPL).
  \item The Decibel Scale is a logarithmic scale. 
  \item For human hearing of loudness, a \enquote{referencd} SPL is used (\ie
    $p_{\text{ref}}$ is the \textit{lowest} threashold of human hearing)
\end{itemize}

\subsubsection{Example}

60dB SPL is 1,000 times louder than the threshold of human hearing.

Loudness (in dB SPL)

\begin{align*}
&= 20 \times \log_{10}\pqty{1000 p_{\text{rms}}/p_{\text{ref}}} \\
&= 20 \times \log_{10}\pqty{1000} \\
&= 20 \times 3 \\
&= 60 \text{dB} \\
\end{align*}

\section{Digitizing Sound}

\subsection{Sampling Rate}

Sound is a continuous, analog signal, but we listen to digital sound all the
time. Digital sound is ultimately coded as discrete units, or bits.

The sampling rate (in seconds) is how often you record information. The sampling
rate should be faster than the period of the fastest frequency you would like to
be able to measure (otherwise you will have \textbf{aliasing errors}). 

The \textbf{Nyquist Limit} is the highest frequency without any aliasing, which
is equal to $\frac{1}{2}$ of the sampling rate. Since humans can hear up to
about 20,000 Hz (20 kHz), we set the industry-standard sampling rate to 44.1 kHz.

\subsection{Bit Rate}

Bit rate refers to how many bits are recorded per second, which is the same as
how sensitive the amplitude samples are, and how much data is used to encode the
amplitude at each sample. This rate also must be sufficient, because if the
recording is not sensitive enough, you will get a \textbf{quantization error}.

Based on human hearing, recordings with 16-bits = $2^{16}$ levels is enough.

\subsection{Dynamic Range}

We must choose the correct range for our amplitude values. If the dynamic range
is not large enough, we will get \textbf{clipping} errors. If the range is too
large, \ie the source sound is too soft, we will also get quantization errors.

\section{Software for Digitizing Sound}

We use Praat. The homework is to use Praat to record ourselves singing the ABC
song.

\end{document}
