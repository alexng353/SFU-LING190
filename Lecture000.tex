\documentclass[12pt]{article}
\usepackage{amsmath}
\usepackage{amssymb}
\usepackage{amsthm}
\usepackage{amsfonts}
\usepackage{graphicx}
\usepackage{textcomp}
\usepackage{hyperref}
\usepackage{tikz}
\usepackage{enumitem}
\usepackage{mathtools}
\usepackage{enumitem}
\usepackage{wasysym}
\usepackage{ulem}
\usepackage{xspace}
\usepackage{booktabs}
\usepackage{physics}

\ifPDFTeX % ensure generation of machine readable output
\input{glyphtounicode}
\pdfgentounicode=1
\usepackage[T1]{fontenc}
\usepackage[utf8]{inputenc}
\usepackage{lmodern}
\fi

\usepackage{csquotes}

\DeclareMathOperator{\dist}{dist}
\DeclareMathOperator{\Nul}{Nul}
\DeclareMathOperator{\Row}{Row}
\DeclareMathOperator{\proj}{proj}

\setlength{\arraycolsep}{12pt}

\newcommand{\Eg}{\textbf{Eg.}\xspace}
\newcommand{\Ex}{\textbf{Ex.}\xspace}
\newcommand{\Ie}{\textbf{I.e.}\xspace}
\newcommand{\bigEps}{\mathcal{E}}
\newcommand{\bproof}{\textit{Proof ($\impliedby$).}\xspace}
\newcommand{\defn}{\textbf{Def.}\xspace}
\newcommand{\eg}{\textbf{e.g.}\xspace}
\newcommand{\ex}{\textbf{ex.}\xspace}
\newcommand{\fproof}{\textit{Proof ($\implies$).}\xspace}
\newcommand{\ie}{\textbf{i.e.}\xspace}
\newcommand{\lemma}{\textit{Lemma}\xspace}
\newcommand{\soln}{\textit{Soln.}\xspace}
\newcommand{\thm}{\textbf{Thm.}\xspace}

\renewcommand{\arraystretch}{1.25} % Adjust row spacing

\hypersetup{
    colorlinks=true,
    linkcolor=blue,
    filecolor=blue,      
    urlcolor=blue,
}

\newcommand{\ulhref}[2]{\href{#1}{\color{blue}\uline{#2}}}

\begin{document}

\title{LING 190 Lecture 0 - Introduction}
\author{Alexander Ng}
\date{May 12, 2025}

\maketitle

\section{Introduction}

\subsection{Why should you care?}

\begin{itemize}
  \item Speech carries lots of information
  \item We are living in the information age
  \item Know the \textit{medium} of your information
\end{itemize}

Now, because of new technologies that allow for high-quality deepfakes, we
should be concerned about the \textit{source} of our information. Both speech
and video deepfakes exist.

\subsection{Who is your instructor?}

Henny Yeung, an Associate Professor in Linguistics. He is a doctor in 
Philosophy(huh?)

\section{Important Information}

\subsection{Office Hours}
Wednesday 10AM - 11AM on Zoom, RCB 9204, or by appointment.

\subsection{Communication}
Use the \textbf{Canvas Messaging System} to communicate with both the instructor
and TAs.

\subsection{Course Structure Information}

\begin{itemize}
  \item Online Lectures are Asynchronous Video Lectures (watch them before the
    next class)
  \item Course engagement is measured by:
    \begin{enumerate}
      \item Intro discussion
      \item Syllabus Quiz based on the syllabus (due Wednesday, April 14)
      \item End-of-course survey (due at end of course)
    \end{enumerate}
  \item Homework - 2.4\% each, 12\% total
    \begin{itemize}
      \item 5 assignments due at the beginning of each class, consisting of 
        practice problems for each quiz.
      \item Open-note
      \item Group-work encouraged
    \end{itemize}
  \item In-class Quizzes - 9-12\% each
    \begin{itemize}
      \item Cumulative (There may be 1 or 2 questions from previous lectures)
      \item Done in-person at the beginning of each class (closed-notes)
      \item Problems are similar to ones encountered in the homework sets.
    \end{itemize}
  \item Final Exam (June 23rd, 10:30AM - 12:20PM, room pending) - 30\%
  \item Final Project (June 30th)
    \begin{itemize}
      \item Group Option:
        \begin{itemize}
          \item (3-5 People)
          \item Narrated Presentation
        \end{itemize}
      \item Individual Option:
        \begin{itemize}
          \item Short literature review (5 pages)
        \end{itemize}
    \end{itemize}
  \item Bonus Group Quiz
    \begin{itemize}
      \item Group quizz - repeat of the same in-class quiz, except done in
        groups of 2-3 people, which count as bonus marks (up to 3\% over the
        entire course).
    \end{itemize}
\end{itemize}

\section{Why should I take this class}

We will be learning about the IPA, how to describe speech sounds, how to improve
evaluations of claims made about speech and understanding how knowing about the 
science of speech can help you.

\begin{enumerate}
  \item Basic Phonetics
    \begin{enumerate}
      \item Acoustics of speech
      \item Mechanisms of the Vocal Tract
        We have extremely sophisticated control over the vocal tract, especially
        the lips, tongue, and throat, which is what allows for speech in the
        first place.
      \item Phonetic Descriptions (International Phonetic Alphabet)
        \begin{itemize}
          \item How to describe sounds using anatomical landmarks
        \end{itemize}
    \end{enumerate}
  \item 
  \item Evaluating Claims about Speech Science
    \begin{enumerate}
      \item Can you change your accent?
        Accent coaching or accent therapy (for actors, etc.)
      \item At what age do you start to learn certain speech sounds?
      \item Can you \enquote{cure} stuttering? How would you do this?
      \item Is there such a thing as a \enquote{voice print} - or unique
        identifier - of your voice?
    \end{enumerate}
  \item Applying speech science in everyday life
    \begin{enumerate}
      \item Apply content knowledge to solve problems
        \begin{itemize}
          \item Environmental noise
          \item How to write \enquote{pronunciation}
          \item How to safeguard your voice
        \end{itemize}
      \item Acquire the tools necessary to further independent study in speech
        science
    \end{enumerate}
\end{enumerate}

\end{document}
